%=+=+=+=+=+=+=+=+=+=+=+=+=+=+=+=+=+=+=+=+=+=+=+=+=+=+=+=+=+=+=+=+=+=+=+=+=+=+=+=
%             _             _
%            | |           | |
%  _ __ ___  | |__    ___  | | ___   ___   _ __       ___   ___   _ __ ___
% | '_ ` _ \ | '_ \  / _ \ | |/ __| / _ \ | '_ \     / __| / _ \ | '_ ` _ \
% | | | | | || | | || (_) || |\__ \| (_) || | | | _ | (__ | (_) || | | | | |
% |_| |_| |_||_| |_| \___/ |_||___/ \___/ |_| |_|(_) \___| \___/ |_| |_| |_|
%
% Author: Mark H. Olson
% Website: https://mholson.com
% Github: https://github.com/mholson
%
% Created: 2021-09-13
%=+=+=+=+=+=+=+=+=+=+=+=+=+=+=+=+=+=+=+=+=+=+=+=+=+=+=+=+=+=+=+=+=+=+=+=+=+=+=+=

%=-=-=-=-=-=-=-=-=-=-=-=-=-=-=-=-=-=-=-=-=-=-=-=-=-=-=-=-=-=-=-=-=-=-=-=-=-=-=-=
% DOCUMENT CLASS & PACKAGES
%=-=-=-=-=-=-=-=-=-=-=-=-=-=-=-=-=-=-=-=-=-=-=-=-=-=-=-=-=-=-=-=-=-=-=-=-=-=-=-=

\documentclass[a4paper, 12pt]{article}

\usepackage{mhocolorthemenord}
\usepackage{mhomath}
\usepackage{mhotikz}
\usepackage{mhotables}
\usepackage{mhoworksheet}

% > > > SETUP THE MARGINS

\addtolength{\oddsidemargin}{1in}
\addtolength{\evensidemargin}{1in}
\addtolength{\textwidth}{-2in}
%\addtolength{\topmargin}{-1.25in}
%\addtolength{\textheight}{2.5in}

% > > > Code From
%  https://tex.stackexchange.com/questions/37707/triangular-numbers-in-tikz
\newlength\radius
\pgfmathsetlength{\radius}{0.5cm}
\newcommand\drawballs[2][]{%
    \foreach \y [evaluate=\y as \yy using #2+1-\y] in {1,...,#2} {%
        \foreach \x in {1,...,\yy} {%
            \shade[shading=ball,ball color=\cnBlue,#1] 
                ({(2*\x-2+\y)*\radius},{sqrt(3)*\y*\radius}) circle (\radius); 
        };
    }%
}

%=-=-=-=-=-=-=-=-=-=-=-=-=-=-=-=-=-=-=-=-=-=-=-=-=-=-=-=-=-=-=-=-=-=-=-=-=-=-=-=
% BEGIN DOCUMENT
%=-=-=-=-=-=-=-=-=-=-=-=-=-=-=-=-=-=-=-=-=-=-=-=-=-=-=-=-=-=-=-=-=-=-=-=-=-=-=-=

\begin{document}

\maketitle % Print the title section

%=-=-=-=-=-=-=-=-=-=-=-=-=-=-=-=-=-=-=-=-=-=-=-=-=-=-=-=-=-=-=-=-=-=-=-=-=-=-=-=
%   SAGE SILENT
%=-=-=-=-=-=-=-=-=-=-=-=-=-=-=-=-=-=-=-=-=-=-=-=-=-=-=-=-=-=-=-=-=-=-=-=-=-=-=-=



%=-=-=-=-=-=-=-=-=-=-=-=-=-=-=-=-=-=-=-=-=-=-=-=-=-=-=-=-=-=-=-=-=-=-=-=-=-=-=-=
%   WORKSHEET INSTRUCTIONS
%=-=-=-=-=-=-=-=-=-=-=-=-=-=-=-=-=-=-=-=-=-=-=-=-=-=-=-=-=-=-=-=-=-=-=-=-=-=-=-=
\begin{center}
\Huge{\cDarkGrey{Triangular Numbers}}
\end{center}

\vspace{0.5cm}
\begin{center}
\includegraphics{tikz/0033}
\end{center}
\vspace{0.5cm}

If we start with one ball, we can generate an equilateral triangles by adding a 
row of balls below the first ball such that all sides have a length of two balls. 
If we let the variable \( n \) represent the position of the object in a list, 
then ball one would have position \( n = 1 \) and the equilateral triangle with 
side length two would have position \( n = 2 \).  To generate the triangle at 
position \( n = 3 \) in the list, we would add another row of balls to the bottom
of the triangle such that each side of the triangle would have a length of three 
balls.

\begin{enumerate}
    \item What is the side length of the equilateral triangle located at position 
    \( n = 4 \) in the list?
    \item What is the side length of the equilateral triangle located as position 
    \( n = 5 \) in the list?  \emph{It might be helpful to draw the triangle}.
    \item What does the value of the position, \( n \), in the list tell 
    us about the triangle?
    \item Each object in the list that is located at position \( n \) is made up 
    \( T_n \) balls.  For example, ball at position \( n = 2 \) is made up of 
    \( T_2 = 3 \) balls.  \( T_n \) is a special list of numbers that we will call 
    the \defw{sequence} of triangular numbers.  Complete the table below. \emph{It might be helpful 
    draw each triangle}.
    \begin{center}
        \(
        \begin{tblr}
        {
          colspec={| c | c | c | c | c | c | c | c | c |},
          width = 0.9\linewidth,
          row{odd} = {nordFour},
          row{even} = {nordSix},
          rowhead = 1,
          rowfoot = 0,
        }
        \hline
         \SetRow{nordNine} n & 1 & 2 & 3 & 4 & 5 & 6 & 7 & 8\\
        \hline
          T_n & 1 & 3 & 6 & \phantom{999} & \phantom{999} & \phantom{999} & \phantom{999} & \phantom{999} \\
            \hline
          \end{tblr}
          \)
        \end{center}
          \item There is a formula for calculating the number of balls that make up 
          the triangle, \( T_n \), for any position \( n \) in the triangular number
          sequence:
          \[
            T_n = \dfrac{n (n + 1)}{2}.  
          \]
          Can you explain using a picture how this formula is working?  
          Hint:  You might want to consider the area of a triangle formula: 
          \[ 
              A_{\triangle} = \dfrac{\text{base} \times \text{height}}{2}. 
            \]

\end{enumerate}

\customfootnosage
\end{document}
